\section{Der RSH-Extraktor}

Zufall findet in vielen Bereichen Anwendung. Simulationen in der Physik, Kryptographie, Kommunikationstechnik und Computerspiele sind nur einige davon. F"ur die meisten Anwendungen reicht sogenannter schwacher, nicht perfekter, Zufall. Dieser l"asst sich leicht und schnell erzeugen, ist jedoch deterministisch und damit prinzipiell vorhersehbar. F"ur sensible Bereiche, wie die Kryptographie, reicht dies nicht aus, da die Sicherheit direkt von dem verwendeten Schl"ussel abh"angt und ist dieser Schl"ussel nicht perfekt zuf"allig, kann er leichter erraten werden. Schnell und viel guten Zufall zu erzeugen ist teuer, deshalb gibt es sogenannte Zufallsextraktoren die aus einer gro"sen Menge schwachen Zufalls und einer kleinen Menge sehr guten Zufalls, guten Zufall extrahiert.

Der \emph{RSH-Extraktor} wurde von Ta-Shma \emph{et al.} \cite{ta2004extractor} entwickelt. Renner \cite{renner2008security} hat bewiesen, dass $2$-$universal$ Hashfunktionen gute Extraktoren sind. Tomamichel \emph{et al.} \cite{tomamichel2011leftover} haben gezeigt, dass das gleiche auch f"ur $\delta$-$almost\ 2$-$universal$ Hashfunktionen gilt, wenn das $\delta$ klein genug ist.

Die Konstruktion des Extraktors besteht aus der Konkatenation von zwei Hash Funktionen und erh"alt als Eingabe einen Seed der L"ange $2l$ (Die genaue Definition von $l$ ist in \cite{mauerer2012modular} Kapitel III.C.3 gegeben). Die erste Hashfunktion ist auch bekannt als \emph{polynomielles Hashing}, bzw. als \emph{Reed-Solomon Code}, da die Konkatenation der Hashes aller Seeds mit der Kodierung durch einen \emph{Reed-Solomon Code} korreliert. Der Eingabestring wird in Bl"ocke der L"ange $l$ zerlegt und jeder Block wird als Element eines Feldes $x_i \in\ $GF$(2^l)$ betrachtet mit dem das Polynom
\begin{equation}
	p_\alpha(x) = \sum_{i=1}^{s} x_i \alpha ^{s-i}
\end{equation}
ausgewertet wird, wobei $\alpha \in\ $GF$(2^l)$ die erste H"alfte des Seeds ist.

Da das $\delta$ des \emph{polynomiellen Hashings} zu gro"s w"are um einen Extraktor zu konstruieren, wurde es mit einer zweiten Hash Funktion kombiniert. Diese wird auch als \emph{Hadamard Code} bezeichnet, da die Verkettung der Ausgabe "uber alle Seeds einer Hadamard-Kodierung entspricht. Die Funktion berechnet die Parit"at des bitweisen Produkts von $p_\alpha(x)$ und der zweiten H"alfte des Seeds. Die Parit"at ist das extrahierte Zufallsbit.

De \emph{et al.} \cite{de2012trevisan} haben gezeigt, dass diese Konstruktion ein \emph{quantum-proof $(4 \log \frac{1}{\varepsilon} + 2$, $2\varepsilon)$-strong extractor} ist.

Mauerer \emph{et. al.} \cite{mauerer2012modular} haben mit \emph{libtrevisan}\footnote{https://github.com/wolfgangmauerer/libtrevisan} eine konkrete Implementierung von Trevisans Extraktor-Konstruktion \cite{trevisan2001extractors} geschrieben. In Ihrem Paper rechnen Sie -- im Gegensatz du De \emph{et al.} -- nicht mit der $\mathcal O$-Notation, sondern betrachten alle Konstanten Faktoren. Der Extraktor ist asymptotisch optimal, jedoch haben die Konstanten Faktoren Einfluss auf die praktische Anwendung, welche der Fokus des Papers war. Dar"uber hinaus haben Sie einen Beweis und eine Implementierung f"ur das \emph{Blockweakdesign} geliefert.
