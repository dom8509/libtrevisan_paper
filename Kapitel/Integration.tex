\section{Integration in libtrevisan}

Libtrevisan ist ein in C++ implementiertes Framework dessen Sourcecode auf GitHub\footnote{https://github.com/wolfgangmauerer/libtrevisan} verf"ugbar ist. Der Name leitet sich aus dem Ziel des Projektes ab, eine Implementierung anzubieten, die Trevisans Extraktor-Konstruktion umsetzt und leicht in eigene Projekte zu integrieren ist. "Uber das \emph{makefile} kann die Verwendung von z.B. NTL\footnote{http://www.shoup.net/ntl/} oder \emph{SSE4}, je nach vorhandener Hard-/Software konfiguriert werden. Durch Verwendung einer Weakdesign und einer Bitextraktor Oberklasse, von denen konkrete Implementierungen erben, ergibt sich ein modularer Aufbau, der leicht um eigene Implementierungen erweitert werden kann.

Eine zus"atzliche Extraktorklasse "ubernimmt den Aufruf des GPU-Algorithmus. Um jedoch die GPU-Implementierung erfolgreich in libtrevisan zu integrieren, musste die Bibliothek an einigen Stellen angepasst werden. Die Struktur ist rein auf eine Ausf"uhrung auf der CPU ausgelegt und um mehrere CPU Kerne zu benutzen werden schlicht mehrere Threads gestartet, welche jeweils ein Bit extrahieren. Die Implementation der GPU-Variante parallelisiert jedoch die einzelne Extraktion und (noch) nicht den kompletten Vorgang. Deshalb wird, wenn erkannt wird, dass die GPU benutzt werden soll, die Anzahl der CPU Threads auf 1 begrenzt. Dar"uberhinaus ist ein h"aufiger Kontextwechsel von CPU zu GPU und umgekehrt von Nachteil, was jedoch ohne "Anderung an libtrevisan unumg"anglich w"are, da bei jedem Extraktionsschritt lokale Daten berechnet werden, die in den Speicher der GPU "ubertragen werden und, wenn ein Bit extrahiert wurde, wieder zur"uck. Um diesen Flaschenhals zu umgehen, werden alle im Laufe der Extraktionen ben"otigten Werte vorher berechnet, in den Speicher der GPU geschrieben und, wenn alle Bits extrahiert wurden, gesammelt wieder zur"uck in den Hauptspeicher geschrieben. Um dies zu erm"oglichen musste die Extraktorklasse um zwei Methoden erweitert werden, wodurch der Extraktoraufruf ebenfalls angepasst werden musste, da diese Methoden nicht in der Oberklasse vorhanden sind und im normalen Extraktionsprozess nicht vorkommen.


Ein zus"atzlicher Parameter im \emph{makefile} bestimmt beim "Ubersetzen von libtrevisan, ob die Cuda-Variante des RSH-Extraktors mit kompiliert wird oder nicht.